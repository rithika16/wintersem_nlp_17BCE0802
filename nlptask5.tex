\documentclass[11pt]{article}

    \usepackage[breakable]{tcolorbox}
    \usepackage{parskip} % Stop auto-indenting (to mimic markdown behaviour)
    
    \usepackage{iftex}
    \ifPDFTeX
    	\usepackage[T1]{fontenc}
    	\usepackage{mathpazo}
    \else
    	\usepackage{fontspec}
    \fi

    % Basic figure setup, for now with no caption control since it's done
    % automatically by Pandoc (which extracts ![](path) syntax from Markdown).
    \usepackage{graphicx}
    % Maintain compatibility with old templates. Remove in nbconvert 6.0
    \let\Oldincludegraphics\includegraphics
    % Ensure that by default, figures have no caption (until we provide a
    % proper Figure object with a Caption API and a way to capture that
    % in the conversion process - todo).
    \usepackage{caption}
    \DeclareCaptionFormat{nocaption}{}
    \captionsetup{format=nocaption,aboveskip=0pt,belowskip=0pt}

    \usepackage[Export]{adjustbox} % Used to constrain images to a maximum size
    \adjustboxset{max size={0.9\linewidth}{0.9\paperheight}}
    \usepackage{float}
    \floatplacement{figure}{H} % forces figures to be placed at the correct location
    \usepackage{xcolor} % Allow colors to be defined
    \usepackage{enumerate} % Needed for markdown enumerations to work
    \usepackage{geometry} % Used to adjust the document margins
    \usepackage{amsmath} % Equations
    \usepackage{amssymb} % Equations
    \usepackage{textcomp} % defines textquotesingle
    % Hack from http://tex.stackexchange.com/a/47451/13684:
    \AtBeginDocument{%
        \def\PYZsq{\textquotesingle}% Upright quotes in Pygmentized code
    }
    \usepackage{upquote} % Upright quotes for verbatim code
    \usepackage{eurosym} % defines \euro
    \usepackage[mathletters]{ucs} % Extended unicode (utf-8) support
    \usepackage{fancyvrb} % verbatim replacement that allows latex
    \usepackage{grffile} % extends the file name processing of package graphics 
                         % to support a larger range
    \makeatletter % fix for grffile with XeLaTeX
    \def\Gread@@xetex#1{%
      \IfFileExists{"\Gin@base".bb}%
      {\Gread@eps{\Gin@base.bb}}%
      {\Gread@@xetex@aux#1}%
    }
    \makeatother

    % The hyperref package gives us a pdf with properly built
    % internal navigation ('pdf bookmarks' for the table of contents,
    % internal cross-reference links, web links for URLs, etc.)
    \usepackage{hyperref}
    % The default LaTeX title has an obnoxious amount of whitespace. By default,
    % titling removes some of it. It also provides customization options.
    \usepackage{titling}
    \usepackage{longtable} % longtable support required by pandoc >1.10
    \usepackage{booktabs}  % table support for pandoc > 1.12.2
    \usepackage[inline]{enumitem} % IRkernel/repr support (it uses the enumerate* environment)
    \usepackage[normalem]{ulem} % ulem is needed to support strikethroughs (\sout)
                                % normalem makes italics be italics, not underlines
    \usepackage{mathrsfs}
    

    
    % Colors for the hyperref package
    \definecolor{urlcolor}{rgb}{0,.145,.698}
    \definecolor{linkcolor}{rgb}{.71,0.21,0.01}
    \definecolor{citecolor}{rgb}{.12,.54,.11}

    % ANSI colors
    \definecolor{ansi-black}{HTML}{3E424D}
    \definecolor{ansi-black-intense}{HTML}{282C36}
    \definecolor{ansi-red}{HTML}{E75C58}
    \definecolor{ansi-red-intense}{HTML}{B22B31}
    \definecolor{ansi-green}{HTML}{00A250}
    \definecolor{ansi-green-intense}{HTML}{007427}
    \definecolor{ansi-yellow}{HTML}{DDB62B}
    \definecolor{ansi-yellow-intense}{HTML}{B27D12}
    \definecolor{ansi-blue}{HTML}{208FFB}
    \definecolor{ansi-blue-intense}{HTML}{0065CA}
    \definecolor{ansi-magenta}{HTML}{D160C4}
    \definecolor{ansi-magenta-intense}{HTML}{A03196}
    \definecolor{ansi-cyan}{HTML}{60C6C8}
    \definecolor{ansi-cyan-intense}{HTML}{258F8F}
    \definecolor{ansi-white}{HTML}{C5C1B4}
    \definecolor{ansi-white-intense}{HTML}{A1A6B2}
    \definecolor{ansi-default-inverse-fg}{HTML}{FFFFFF}
    \definecolor{ansi-default-inverse-bg}{HTML}{000000}

    % commands and environments needed by pandoc snippets
    % extracted from the output of `pandoc -s`
    \providecommand{\tightlist}{%
      \setlength{\itemsep}{0pt}\setlength{\parskip}{0pt}}
    \DefineVerbatimEnvironment{Highlighting}{Verbatim}{commandchars=\\\{\}}
    % Add ',fontsize=\small' for more characters per line
    \newenvironment{Shaded}{}{}
    \newcommand{\KeywordTok}[1]{\textcolor[rgb]{0.00,0.44,0.13}{\textbf{{#1}}}}
    \newcommand{\DataTypeTok}[1]{\textcolor[rgb]{0.56,0.13,0.00}{{#1}}}
    \newcommand{\DecValTok}[1]{\textcolor[rgb]{0.25,0.63,0.44}{{#1}}}
    \newcommand{\BaseNTok}[1]{\textcolor[rgb]{0.25,0.63,0.44}{{#1}}}
    \newcommand{\FloatTok}[1]{\textcolor[rgb]{0.25,0.63,0.44}{{#1}}}
    \newcommand{\CharTok}[1]{\textcolor[rgb]{0.25,0.44,0.63}{{#1}}}
    \newcommand{\StringTok}[1]{\textcolor[rgb]{0.25,0.44,0.63}{{#1}}}
    \newcommand{\CommentTok}[1]{\textcolor[rgb]{0.38,0.63,0.69}{\textit{{#1}}}}
    \newcommand{\OtherTok}[1]{\textcolor[rgb]{0.00,0.44,0.13}{{#1}}}
    \newcommand{\AlertTok}[1]{\textcolor[rgb]{1.00,0.00,0.00}{\textbf{{#1}}}}
    \newcommand{\FunctionTok}[1]{\textcolor[rgb]{0.02,0.16,0.49}{{#1}}}
    \newcommand{\RegionMarkerTok}[1]{{#1}}
    \newcommand{\ErrorTok}[1]{\textcolor[rgb]{1.00,0.00,0.00}{\textbf{{#1}}}}
    \newcommand{\NormalTok}[1]{{#1}}
    
    % Additional commands for more recent versions of Pandoc
    \newcommand{\ConstantTok}[1]{\textcolor[rgb]{0.53,0.00,0.00}{{#1}}}
    \newcommand{\SpecialCharTok}[1]{\textcolor[rgb]{0.25,0.44,0.63}{{#1}}}
    \newcommand{\VerbatimStringTok}[1]{\textcolor[rgb]{0.25,0.44,0.63}{{#1}}}
    \newcommand{\SpecialStringTok}[1]{\textcolor[rgb]{0.73,0.40,0.53}{{#1}}}
    \newcommand{\ImportTok}[1]{{#1}}
    \newcommand{\DocumentationTok}[1]{\textcolor[rgb]{0.73,0.13,0.13}{\textit{{#1}}}}
    \newcommand{\AnnotationTok}[1]{\textcolor[rgb]{0.38,0.63,0.69}{\textbf{\textit{{#1}}}}}
    \newcommand{\CommentVarTok}[1]{\textcolor[rgb]{0.38,0.63,0.69}{\textbf{\textit{{#1}}}}}
    \newcommand{\VariableTok}[1]{\textcolor[rgb]{0.10,0.09,0.49}{{#1}}}
    \newcommand{\ControlFlowTok}[1]{\textcolor[rgb]{0.00,0.44,0.13}{\textbf{{#1}}}}
    \newcommand{\OperatorTok}[1]{\textcolor[rgb]{0.40,0.40,0.40}{{#1}}}
    \newcommand{\BuiltInTok}[1]{{#1}}
    \newcommand{\ExtensionTok}[1]{{#1}}
    \newcommand{\PreprocessorTok}[1]{\textcolor[rgb]{0.74,0.48,0.00}{{#1}}}
    \newcommand{\AttributeTok}[1]{\textcolor[rgb]{0.49,0.56,0.16}{{#1}}}
    \newcommand{\InformationTok}[1]{\textcolor[rgb]{0.38,0.63,0.69}{\textbf{\textit{{#1}}}}}
    \newcommand{\WarningTok}[1]{\textcolor[rgb]{0.38,0.63,0.69}{\textbf{\textit{{#1}}}}}
    
    
    % Define a nice break command that doesn't care if a line doesn't already
    % exist.
    \def\br{\hspace*{\fill} \\* }
    % Math Jax compatibility definitions
    \def\gt{>}
    \def\lt{<}
    \let\Oldtex\TeX
    \let\Oldlatex\LaTeX
    \renewcommand{\TeX}{\textrm{\Oldtex}}
    \renewcommand{\LaTeX}{\textrm{\Oldlatex}}
    % Document parameters
    % Document title
    \title{nlptask5}
    
    
    
    
    
% Pygments definitions
\makeatletter
\def\PY@reset{\let\PY@it=\relax \let\PY@bf=\relax%
    \let\PY@ul=\relax \let\PY@tc=\relax%
    \let\PY@bc=\relax \let\PY@ff=\relax}
\def\PY@tok#1{\csname PY@tok@#1\endcsname}
\def\PY@toks#1+{\ifx\relax#1\empty\else%
    \PY@tok{#1}\expandafter\PY@toks\fi}
\def\PY@do#1{\PY@bc{\PY@tc{\PY@ul{%
    \PY@it{\PY@bf{\PY@ff{#1}}}}}}}
\def\PY#1#2{\PY@reset\PY@toks#1+\relax+\PY@do{#2}}

\expandafter\def\csname PY@tok@w\endcsname{\def\PY@tc##1{\textcolor[rgb]{0.73,0.73,0.73}{##1}}}
\expandafter\def\csname PY@tok@c\endcsname{\let\PY@it=\textit\def\PY@tc##1{\textcolor[rgb]{0.25,0.50,0.50}{##1}}}
\expandafter\def\csname PY@tok@cp\endcsname{\def\PY@tc##1{\textcolor[rgb]{0.74,0.48,0.00}{##1}}}
\expandafter\def\csname PY@tok@k\endcsname{\let\PY@bf=\textbf\def\PY@tc##1{\textcolor[rgb]{0.00,0.50,0.00}{##1}}}
\expandafter\def\csname PY@tok@kp\endcsname{\def\PY@tc##1{\textcolor[rgb]{0.00,0.50,0.00}{##1}}}
\expandafter\def\csname PY@tok@kt\endcsname{\def\PY@tc##1{\textcolor[rgb]{0.69,0.00,0.25}{##1}}}
\expandafter\def\csname PY@tok@o\endcsname{\def\PY@tc##1{\textcolor[rgb]{0.40,0.40,0.40}{##1}}}
\expandafter\def\csname PY@tok@ow\endcsname{\let\PY@bf=\textbf\def\PY@tc##1{\textcolor[rgb]{0.67,0.13,1.00}{##1}}}
\expandafter\def\csname PY@tok@nb\endcsname{\def\PY@tc##1{\textcolor[rgb]{0.00,0.50,0.00}{##1}}}
\expandafter\def\csname PY@tok@nf\endcsname{\def\PY@tc##1{\textcolor[rgb]{0.00,0.00,1.00}{##1}}}
\expandafter\def\csname PY@tok@nc\endcsname{\let\PY@bf=\textbf\def\PY@tc##1{\textcolor[rgb]{0.00,0.00,1.00}{##1}}}
\expandafter\def\csname PY@tok@nn\endcsname{\let\PY@bf=\textbf\def\PY@tc##1{\textcolor[rgb]{0.00,0.00,1.00}{##1}}}
\expandafter\def\csname PY@tok@ne\endcsname{\let\PY@bf=\textbf\def\PY@tc##1{\textcolor[rgb]{0.82,0.25,0.23}{##1}}}
\expandafter\def\csname PY@tok@nv\endcsname{\def\PY@tc##1{\textcolor[rgb]{0.10,0.09,0.49}{##1}}}
\expandafter\def\csname PY@tok@no\endcsname{\def\PY@tc##1{\textcolor[rgb]{0.53,0.00,0.00}{##1}}}
\expandafter\def\csname PY@tok@nl\endcsname{\def\PY@tc##1{\textcolor[rgb]{0.63,0.63,0.00}{##1}}}
\expandafter\def\csname PY@tok@ni\endcsname{\let\PY@bf=\textbf\def\PY@tc##1{\textcolor[rgb]{0.60,0.60,0.60}{##1}}}
\expandafter\def\csname PY@tok@na\endcsname{\def\PY@tc##1{\textcolor[rgb]{0.49,0.56,0.16}{##1}}}
\expandafter\def\csname PY@tok@nt\endcsname{\let\PY@bf=\textbf\def\PY@tc##1{\textcolor[rgb]{0.00,0.50,0.00}{##1}}}
\expandafter\def\csname PY@tok@nd\endcsname{\def\PY@tc##1{\textcolor[rgb]{0.67,0.13,1.00}{##1}}}
\expandafter\def\csname PY@tok@s\endcsname{\def\PY@tc##1{\textcolor[rgb]{0.73,0.13,0.13}{##1}}}
\expandafter\def\csname PY@tok@sd\endcsname{\let\PY@it=\textit\def\PY@tc##1{\textcolor[rgb]{0.73,0.13,0.13}{##1}}}
\expandafter\def\csname PY@tok@si\endcsname{\let\PY@bf=\textbf\def\PY@tc##1{\textcolor[rgb]{0.73,0.40,0.53}{##1}}}
\expandafter\def\csname PY@tok@se\endcsname{\let\PY@bf=\textbf\def\PY@tc##1{\textcolor[rgb]{0.73,0.40,0.13}{##1}}}
\expandafter\def\csname PY@tok@sr\endcsname{\def\PY@tc##1{\textcolor[rgb]{0.73,0.40,0.53}{##1}}}
\expandafter\def\csname PY@tok@ss\endcsname{\def\PY@tc##1{\textcolor[rgb]{0.10,0.09,0.49}{##1}}}
\expandafter\def\csname PY@tok@sx\endcsname{\def\PY@tc##1{\textcolor[rgb]{0.00,0.50,0.00}{##1}}}
\expandafter\def\csname PY@tok@m\endcsname{\def\PY@tc##1{\textcolor[rgb]{0.40,0.40,0.40}{##1}}}
\expandafter\def\csname PY@tok@gh\endcsname{\let\PY@bf=\textbf\def\PY@tc##1{\textcolor[rgb]{0.00,0.00,0.50}{##1}}}
\expandafter\def\csname PY@tok@gu\endcsname{\let\PY@bf=\textbf\def\PY@tc##1{\textcolor[rgb]{0.50,0.00,0.50}{##1}}}
\expandafter\def\csname PY@tok@gd\endcsname{\def\PY@tc##1{\textcolor[rgb]{0.63,0.00,0.00}{##1}}}
\expandafter\def\csname PY@tok@gi\endcsname{\def\PY@tc##1{\textcolor[rgb]{0.00,0.63,0.00}{##1}}}
\expandafter\def\csname PY@tok@gr\endcsname{\def\PY@tc##1{\textcolor[rgb]{1.00,0.00,0.00}{##1}}}
\expandafter\def\csname PY@tok@ge\endcsname{\let\PY@it=\textit}
\expandafter\def\csname PY@tok@gs\endcsname{\let\PY@bf=\textbf}
\expandafter\def\csname PY@tok@gp\endcsname{\let\PY@bf=\textbf\def\PY@tc##1{\textcolor[rgb]{0.00,0.00,0.50}{##1}}}
\expandafter\def\csname PY@tok@go\endcsname{\def\PY@tc##1{\textcolor[rgb]{0.53,0.53,0.53}{##1}}}
\expandafter\def\csname PY@tok@gt\endcsname{\def\PY@tc##1{\textcolor[rgb]{0.00,0.27,0.87}{##1}}}
\expandafter\def\csname PY@tok@err\endcsname{\def\PY@bc##1{\setlength{\fboxsep}{0pt}\fcolorbox[rgb]{1.00,0.00,0.00}{1,1,1}{\strut ##1}}}
\expandafter\def\csname PY@tok@kc\endcsname{\let\PY@bf=\textbf\def\PY@tc##1{\textcolor[rgb]{0.00,0.50,0.00}{##1}}}
\expandafter\def\csname PY@tok@kd\endcsname{\let\PY@bf=\textbf\def\PY@tc##1{\textcolor[rgb]{0.00,0.50,0.00}{##1}}}
\expandafter\def\csname PY@tok@kn\endcsname{\let\PY@bf=\textbf\def\PY@tc##1{\textcolor[rgb]{0.00,0.50,0.00}{##1}}}
\expandafter\def\csname PY@tok@kr\endcsname{\let\PY@bf=\textbf\def\PY@tc##1{\textcolor[rgb]{0.00,0.50,0.00}{##1}}}
\expandafter\def\csname PY@tok@bp\endcsname{\def\PY@tc##1{\textcolor[rgb]{0.00,0.50,0.00}{##1}}}
\expandafter\def\csname PY@tok@fm\endcsname{\def\PY@tc##1{\textcolor[rgb]{0.00,0.00,1.00}{##1}}}
\expandafter\def\csname PY@tok@vc\endcsname{\def\PY@tc##1{\textcolor[rgb]{0.10,0.09,0.49}{##1}}}
\expandafter\def\csname PY@tok@vg\endcsname{\def\PY@tc##1{\textcolor[rgb]{0.10,0.09,0.49}{##1}}}
\expandafter\def\csname PY@tok@vi\endcsname{\def\PY@tc##1{\textcolor[rgb]{0.10,0.09,0.49}{##1}}}
\expandafter\def\csname PY@tok@vm\endcsname{\def\PY@tc##1{\textcolor[rgb]{0.10,0.09,0.49}{##1}}}
\expandafter\def\csname PY@tok@sa\endcsname{\def\PY@tc##1{\textcolor[rgb]{0.73,0.13,0.13}{##1}}}
\expandafter\def\csname PY@tok@sb\endcsname{\def\PY@tc##1{\textcolor[rgb]{0.73,0.13,0.13}{##1}}}
\expandafter\def\csname PY@tok@sc\endcsname{\def\PY@tc##1{\textcolor[rgb]{0.73,0.13,0.13}{##1}}}
\expandafter\def\csname PY@tok@dl\endcsname{\def\PY@tc##1{\textcolor[rgb]{0.73,0.13,0.13}{##1}}}
\expandafter\def\csname PY@tok@s2\endcsname{\def\PY@tc##1{\textcolor[rgb]{0.73,0.13,0.13}{##1}}}
\expandafter\def\csname PY@tok@sh\endcsname{\def\PY@tc##1{\textcolor[rgb]{0.73,0.13,0.13}{##1}}}
\expandafter\def\csname PY@tok@s1\endcsname{\def\PY@tc##1{\textcolor[rgb]{0.73,0.13,0.13}{##1}}}
\expandafter\def\csname PY@tok@mb\endcsname{\def\PY@tc##1{\textcolor[rgb]{0.40,0.40,0.40}{##1}}}
\expandafter\def\csname PY@tok@mf\endcsname{\def\PY@tc##1{\textcolor[rgb]{0.40,0.40,0.40}{##1}}}
\expandafter\def\csname PY@tok@mh\endcsname{\def\PY@tc##1{\textcolor[rgb]{0.40,0.40,0.40}{##1}}}
\expandafter\def\csname PY@tok@mi\endcsname{\def\PY@tc##1{\textcolor[rgb]{0.40,0.40,0.40}{##1}}}
\expandafter\def\csname PY@tok@il\endcsname{\def\PY@tc##1{\textcolor[rgb]{0.40,0.40,0.40}{##1}}}
\expandafter\def\csname PY@tok@mo\endcsname{\def\PY@tc##1{\textcolor[rgb]{0.40,0.40,0.40}{##1}}}
\expandafter\def\csname PY@tok@ch\endcsname{\let\PY@it=\textit\def\PY@tc##1{\textcolor[rgb]{0.25,0.50,0.50}{##1}}}
\expandafter\def\csname PY@tok@cm\endcsname{\let\PY@it=\textit\def\PY@tc##1{\textcolor[rgb]{0.25,0.50,0.50}{##1}}}
\expandafter\def\csname PY@tok@cpf\endcsname{\let\PY@it=\textit\def\PY@tc##1{\textcolor[rgb]{0.25,0.50,0.50}{##1}}}
\expandafter\def\csname PY@tok@c1\endcsname{\let\PY@it=\textit\def\PY@tc##1{\textcolor[rgb]{0.25,0.50,0.50}{##1}}}
\expandafter\def\csname PY@tok@cs\endcsname{\let\PY@it=\textit\def\PY@tc##1{\textcolor[rgb]{0.25,0.50,0.50}{##1}}}

\def\PYZbs{\char`\\}
\def\PYZus{\char`\_}
\def\PYZob{\char`\{}
\def\PYZcb{\char`\}}
\def\PYZca{\char`\^}
\def\PYZam{\char`\&}
\def\PYZlt{\char`\<}
\def\PYZgt{\char`\>}
\def\PYZsh{\char`\#}
\def\PYZpc{\char`\%}
\def\PYZdl{\char`\$}
\def\PYZhy{\char`\-}
\def\PYZsq{\char`\'}
\def\PYZdq{\char`\"}
\def\PYZti{\char`\~}
% for compatibility with earlier versions
\def\PYZat{@}
\def\PYZlb{[}
\def\PYZrb{]}
\makeatother


    % For linebreaks inside Verbatim environment from package fancyvrb. 
    \makeatletter
        \newbox\Wrappedcontinuationbox 
        \newbox\Wrappedvisiblespacebox 
        \newcommand*\Wrappedvisiblespace {\textcolor{red}{\textvisiblespace}} 
        \newcommand*\Wrappedcontinuationsymbol {\textcolor{red}{\llap{\tiny$\m@th\hookrightarrow$}}} 
        \newcommand*\Wrappedcontinuationindent {3ex } 
        \newcommand*\Wrappedafterbreak {\kern\Wrappedcontinuationindent\copy\Wrappedcontinuationbox} 
        % Take advantage of the already applied Pygments mark-up to insert 
        % potential linebreaks for TeX processing. 
        %        {, <, #, %, $, ' and ": go to next line. 
        %        _, }, ^, &, >, - and ~: stay at end of broken line. 
        % Use of \textquotesingle for straight quote. 
        \newcommand*\Wrappedbreaksatspecials {% 
            \def\PYGZus{\discretionary{\char`\_}{\Wrappedafterbreak}{\char`\_}}% 
            \def\PYGZob{\discretionary{}{\Wrappedafterbreak\char`\{}{\char`\{}}% 
            \def\PYGZcb{\discretionary{\char`\}}{\Wrappedafterbreak}{\char`\}}}% 
            \def\PYGZca{\discretionary{\char`\^}{\Wrappedafterbreak}{\char`\^}}% 
            \def\PYGZam{\discretionary{\char`\&}{\Wrappedafterbreak}{\char`\&}}% 
            \def\PYGZlt{\discretionary{}{\Wrappedafterbreak\char`\<}{\char`\<}}% 
            \def\PYGZgt{\discretionary{\char`\>}{\Wrappedafterbreak}{\char`\>}}% 
            \def\PYGZsh{\discretionary{}{\Wrappedafterbreak\char`\#}{\char`\#}}% 
            \def\PYGZpc{\discretionary{}{\Wrappedafterbreak\char`\%}{\char`\%}}% 
            \def\PYGZdl{\discretionary{}{\Wrappedafterbreak\char`\$}{\char`\$}}% 
            \def\PYGZhy{\discretionary{\char`\-}{\Wrappedafterbreak}{\char`\-}}% 
            \def\PYGZsq{\discretionary{}{\Wrappedafterbreak\textquotesingle}{\textquotesingle}}% 
            \def\PYGZdq{\discretionary{}{\Wrappedafterbreak\char`\"}{\char`\"}}% 
            \def\PYGZti{\discretionary{\char`\~}{\Wrappedafterbreak}{\char`\~}}% 
        } 
        % Some characters . , ; ? ! / are not pygmentized. 
        % This macro makes them "active" and they will insert potential linebreaks 
        \newcommand*\Wrappedbreaksatpunct {% 
            \lccode`\~`\.\lowercase{\def~}{\discretionary{\hbox{\char`\.}}{\Wrappedafterbreak}{\hbox{\char`\.}}}% 
            \lccode`\~`\,\lowercase{\def~}{\discretionary{\hbox{\char`\,}}{\Wrappedafterbreak}{\hbox{\char`\,}}}% 
            \lccode`\~`\;\lowercase{\def~}{\discretionary{\hbox{\char`\;}}{\Wrappedafterbreak}{\hbox{\char`\;}}}% 
            \lccode`\~`\:\lowercase{\def~}{\discretionary{\hbox{\char`\:}}{\Wrappedafterbreak}{\hbox{\char`\:}}}% 
            \lccode`\~`\?\lowercase{\def~}{\discretionary{\hbox{\char`\?}}{\Wrappedafterbreak}{\hbox{\char`\?}}}% 
            \lccode`\~`\!\lowercase{\def~}{\discretionary{\hbox{\char`\!}}{\Wrappedafterbreak}{\hbox{\char`\!}}}% 
            \lccode`\~`\/\lowercase{\def~}{\discretionary{\hbox{\char`\/}}{\Wrappedafterbreak}{\hbox{\char`\/}}}% 
            \catcode`\.\active
            \catcode`\,\active 
            \catcode`\;\active
            \catcode`\:\active
            \catcode`\?\active
            \catcode`\!\active
            \catcode`\/\active 
            \lccode`\~`\~ 	
        }
    \makeatother

    \let\OriginalVerbatim=\Verbatim
    \makeatletter
    \renewcommand{\Verbatim}[1][1]{%
        %\parskip\z@skip
        \sbox\Wrappedcontinuationbox {\Wrappedcontinuationsymbol}%
        \sbox\Wrappedvisiblespacebox {\FV@SetupFont\Wrappedvisiblespace}%
        \def\FancyVerbFormatLine ##1{\hsize\linewidth
            \vtop{\raggedright\hyphenpenalty\z@\exhyphenpenalty\z@
                \doublehyphendemerits\z@\finalhyphendemerits\z@
                \strut ##1\strut}%
        }%
        % If the linebreak is at a space, the latter will be displayed as visible
        % space at end of first line, and a continuation symbol starts next line.
        % Stretch/shrink are however usually zero for typewriter font.
        \def\FV@Space {%
            \nobreak\hskip\z@ plus\fontdimen3\font minus\fontdimen4\font
            \discretionary{\copy\Wrappedvisiblespacebox}{\Wrappedafterbreak}
            {\kern\fontdimen2\font}%
        }%
        
        % Allow breaks at special characters using \PYG... macros.
        \Wrappedbreaksatspecials
        % Breaks at punctuation characters . , ; ? ! and / need catcode=\active 	
        \OriginalVerbatim[#1,codes*=\Wrappedbreaksatpunct]%
    }
    \makeatother

    % Exact colors from NB
    \definecolor{incolor}{HTML}{303F9F}
    \definecolor{outcolor}{HTML}{D84315}
    \definecolor{cellborder}{HTML}{CFCFCF}
    \definecolor{cellbackground}{HTML}{F7F7F7}
    
    % prompt
    \makeatletter
    \newcommand{\boxspacing}{\kern\kvtcb@left@rule\kern\kvtcb@boxsep}
    \makeatother
    \newcommand{\prompt}[4]{
        \ttfamily\llap{{\color{#2}[#3]:\hspace{3pt}#4}}\vspace{-\baselineskip}
    }
    

    
    % Prevent overflowing lines due to hard-to-break entities
    \sloppy 
    % Setup hyperref package
    \hypersetup{
      breaklinks=true,  % so long urls are correctly broken across lines
      colorlinks=true,
      urlcolor=urlcolor,
      linkcolor=linkcolor,
      citecolor=citecolor,
      }
    % Slightly bigger margins than the latex defaults
    
    \geometry{verbose,tmargin=1in,bmargin=1in,lmargin=1in,rmargin=1in}
    
    

\begin{document}
    
    \maketitle
    
    

    
    \begin{tcolorbox}[breakable, size=fbox, boxrule=1pt, pad at break*=1mm,colback=cellbackground, colframe=cellborder]
\prompt{In}{incolor}{2}{\boxspacing}
\begin{Verbatim}[commandchars=\\\{\}]
\PY{k+kn}{from} \PY{n+nn}{nltk}\PY{n+nn}{.}\PY{n+nn}{corpus} \PY{k}{import} \PY{n}{stopwords}
\PY{n}{stopwords}\PY{o}{.}\PY{n}{words}\PY{p}{(}\PY{l+s+s1}{\PYZsq{}}\PY{l+s+s1}{english}\PY{l+s+s1}{\PYZsq{}}\PY{p}{)}
\end{Verbatim}
\end{tcolorbox}

            \begin{tcolorbox}[breakable, size=fbox, boxrule=.5pt, pad at break*=1mm, opacityfill=0]
\prompt{Out}{outcolor}{2}{\boxspacing}
\begin{Verbatim}[commandchars=\\\{\}]
['i',
 'me',
 'my',
 'myself',
 'we',
 'our',
 'ours',
 'ourselves',
 'you',
 "you're",
 "you've",
 "you'll",
 "you'd",
 'your',
 'yours',
 'yourself',
 'yourselves',
 'he',
 'him',
 'his',
 'himself',
 'she',
 "she's",
 'her',
 'hers',
 'herself',
 'it',
 "it's",
 'its',
 'itself',
 'they',
 'them',
 'their',
 'theirs',
 'themselves',
 'what',
 'which',
 'who',
 'whom',
 'this',
 'that',
 "that'll",
 'these',
 'those',
 'am',
 'is',
 'are',
 'was',
 'were',
 'be',
 'been',
 'being',
 'have',
 'has',
 'had',
 'having',
 'do',
 'does',
 'did',
 'doing',
 'a',
 'an',
 'the',
 'and',
 'but',
 'if',
 'or',
 'because',
 'as',
 'until',
 'while',
 'of',
 'at',
 'by',
 'for',
 'with',
 'about',
 'against',
 'between',
 'into',
 'through',
 'during',
 'before',
 'after',
 'above',
 'below',
 'to',
 'from',
 'up',
 'down',
 'in',
 'out',
 'on',
 'off',
 'over',
 'under',
 'again',
 'further',
 'then',
 'once',
 'here',
 'there',
 'when',
 'where',
 'why',
 'how',
 'all',
 'any',
 'both',
 'each',
 'few',
 'more',
 'most',
 'other',
 'some',
 'such',
 'no',
 'nor',
 'not',
 'only',
 'own',
 'same',
 'so',
 'than',
 'too',
 'very',
 's',
 't',
 'can',
 'will',
 'just',
 'don',
 "don't",
 'should',
 "should've",
 'now',
 'd',
 'll',
 'm',
 'o',
 're',
 've',
 'y',
 'ain',
 'aren',
 "aren't",
 'couldn',
 "couldn't",
 'didn',
 "didn't",
 'doesn',
 "doesn't",
 'hadn',
 "hadn't",
 'hasn',
 "hasn't",
 'haven',
 "haven't",
 'isn',
 "isn't",
 'ma',
 'mightn',
 "mightn't",
 'mustn',
 "mustn't",
 'needn',
 "needn't",
 'shan',
 "shan't",
 'shouldn',
 "shouldn't",
 'wasn',
 "wasn't",
 'weren',
 "weren't",
 'won',
 "won't",
 'wouldn',
 "wouldn't"]
\end{Verbatim}
\end{tcolorbox}
        
    \begin{tcolorbox}[breakable, size=fbox, boxrule=1pt, pad at break*=1mm,colback=cellbackground, colframe=cellborder]
\prompt{In}{incolor}{3}{\boxspacing}
\begin{Verbatim}[commandchars=\\\{\}]
\PY{k+kn}{from} \PY{n+nn}{nltk}\PY{n+nn}{.}\PY{n+nn}{corpus} \PY{k}{import} \PY{n}{stopwords}
\PY{n}{stopwords}\PY{o}{.}\PY{n}{words}\PY{p}{(}\PY{l+s+s1}{\PYZsq{}}\PY{l+s+s1}{French}\PY{l+s+s1}{\PYZsq{}}\PY{p}{)}
\end{Verbatim}
\end{tcolorbox}

            \begin{tcolorbox}[breakable, size=fbox, boxrule=.5pt, pad at break*=1mm, opacityfill=0]
\prompt{Out}{outcolor}{3}{\boxspacing}
\begin{Verbatim}[commandchars=\\\{\}]
['au',
 'aux',
 'avec',
 'ce',
 'ces',
 'dans',
 'de',
 'des',
 'du',
 'elle',
 'en',
 'et',
 'eux',
 'il',
 'ils',
 'je',
 'la',
 'le',
 'les',
 'leur',
 'lui',
 'ma',
 'mais',
 'me',
 'même',
 'mes',
 'moi',
 'mon',
 'ne',
 'nos',
 'notre',
 'nous',
 'on',
 'ou',
 'par',
 'pas',
 'pour',
 'qu',
 'que',
 'qui',
 'sa',
 'se',
 'ses',
 'son',
 'sur',
 'ta',
 'te',
 'tes',
 'toi',
 'ton',
 'tu',
 'un',
 'une',
 'vos',
 'votre',
 'vous',
 'c',
 'd',
 'j',
 'l',
 'à',
 'm',
 'n',
 's',
 't',
 'y',
 'été',
 'étée',
 'étées',
 'étés',
 'étant',
 'étante',
 'étants',
 'étantes',
 'suis',
 'es',
 'est',
 'sommes',
 'êtes',
 'sont',
 'serai',
 'seras',
 'sera',
 'serons',
 'serez',
 'seront',
 'serais',
 'serait',
 'serions',
 'seriez',
 'seraient',
 'étais',
 'était',
 'étions',
 'étiez',
 'étaient',
 'fus',
 'fut',
 'fûmes',
 'fûtes',
 'furent',
 'sois',
 'soit',
 'soyons',
 'soyez',
 'soient',
 'fusse',
 'fusses',
 'fût',
 'fussions',
 'fussiez',
 'fussent',
 'ayant',
 'ayante',
 'ayantes',
 'ayants',
 'eu',
 'eue',
 'eues',
 'eus',
 'ai',
 'as',
 'avons',
 'avez',
 'ont',
 'aurai',
 'auras',
 'aura',
 'aurons',
 'aurez',
 'auront',
 'aurais',
 'aurait',
 'aurions',
 'auriez',
 'auraient',
 'avais',
 'avait',
 'avions',
 'aviez',
 'avaient',
 'eut',
 'eûmes',
 'eûtes',
 'eurent',
 'aie',
 'aies',
 'ait',
 'ayons',
 'ayez',
 'aient',
 'eusse',
 'eusses',
 'eût',
 'eussions',
 'eussiez',
 'eussent']
\end{Verbatim}
\end{tcolorbox}
        
    \begin{tcolorbox}[breakable, size=fbox, boxrule=1pt, pad at break*=1mm,colback=cellbackground, colframe=cellborder]
\prompt{In}{incolor}{4}{\boxspacing}
\begin{Verbatim}[commandchars=\\\{\}]
\PY{k+kn}{from} \PY{n+nn}{nltk}\PY{n+nn}{.}\PY{n+nn}{corpus} \PY{k}{import} \PY{n}{stopwords}
\PY{n}{stopwords}\PY{o}{.}\PY{n}{words}\PY{p}{(}\PY{l+s+s1}{\PYZsq{}}\PY{l+s+s1}{german}\PY{l+s+s1}{\PYZsq{}}\PY{p}{)}
\end{Verbatim}
\end{tcolorbox}

            \begin{tcolorbox}[breakable, size=fbox, boxrule=.5pt, pad at break*=1mm, opacityfill=0]
\prompt{Out}{outcolor}{4}{\boxspacing}
\begin{Verbatim}[commandchars=\\\{\}]
['aber',
 'alle',
 'allem',
 'allen',
 'aller',
 'alles',
 'als',
 'also',
 'am',
 'an',
 'ander',
 'andere',
 'anderem',
 'anderen',
 'anderer',
 'anderes',
 'anderm',
 'andern',
 'anderr',
 'anders',
 'auch',
 'auf',
 'aus',
 'bei',
 'bin',
 'bis',
 'bist',
 'da',
 'damit',
 'dann',
 'der',
 'den',
 'des',
 'dem',
 'die',
 'das',
 'dass',
 'daß',
 'derselbe',
 'derselben',
 'denselben',
 'desselben',
 'demselben',
 'dieselbe',
 'dieselben',
 'dasselbe',
 'dazu',
 'dein',
 'deine',
 'deinem',
 'deinen',
 'deiner',
 'deines',
 'denn',
 'derer',
 'dessen',
 'dich',
 'dir',
 'du',
 'dies',
 'diese',
 'diesem',
 'diesen',
 'dieser',
 'dieses',
 'doch',
 'dort',
 'durch',
 'ein',
 'eine',
 'einem',
 'einen',
 'einer',
 'eines',
 'einig',
 'einige',
 'einigem',
 'einigen',
 'einiger',
 'einiges',
 'einmal',
 'er',
 'ihn',
 'ihm',
 'es',
 'etwas',
 'euer',
 'eure',
 'eurem',
 'euren',
 'eurer',
 'eures',
 'für',
 'gegen',
 'gewesen',
 'hab',
 'habe',
 'haben',
 'hat',
 'hatte',
 'hatten',
 'hier',
 'hin',
 'hinter',
 'ich',
 'mich',
 'mir',
 'ihr',
 'ihre',
 'ihrem',
 'ihren',
 'ihrer',
 'ihres',
 'euch',
 'im',
 'in',
 'indem',
 'ins',
 'ist',
 'jede',
 'jedem',
 'jeden',
 'jeder',
 'jedes',
 'jene',
 'jenem',
 'jenen',
 'jener',
 'jenes',
 'jetzt',
 'kann',
 'kein',
 'keine',
 'keinem',
 'keinen',
 'keiner',
 'keines',
 'können',
 'könnte',
 'machen',
 'man',
 'manche',
 'manchem',
 'manchen',
 'mancher',
 'manches',
 'mein',
 'meine',
 'meinem',
 'meinen',
 'meiner',
 'meines',
 'mit',
 'muss',
 'musste',
 'nach',
 'nicht',
 'nichts',
 'noch',
 'nun',
 'nur',
 'ob',
 'oder',
 'ohne',
 'sehr',
 'sein',
 'seine',
 'seinem',
 'seinen',
 'seiner',
 'seines',
 'selbst',
 'sich',
 'sie',
 'ihnen',
 'sind',
 'so',
 'solche',
 'solchem',
 'solchen',
 'solcher',
 'solches',
 'soll',
 'sollte',
 'sondern',
 'sonst',
 'über',
 'um',
 'und',
 'uns',
 'unsere',
 'unserem',
 'unseren',
 'unser',
 'unseres',
 'unter',
 'viel',
 'vom',
 'von',
 'vor',
 'während',
 'war',
 'waren',
 'warst',
 'was',
 'weg',
 'weil',
 'weiter',
 'welche',
 'welchem',
 'welchen',
 'welcher',
 'welches',
 'wenn',
 'werde',
 'werden',
 'wie',
 'wieder',
 'will',
 'wir',
 'wird',
 'wirst',
 'wo',
 'wollen',
 'wollte',
 'würde',
 'würden',
 'zu',
 'zum',
 'zur',
 'zwar',
 'zwischen']
\end{Verbatim}
\end{tcolorbox}
        
    \begin{tcolorbox}[breakable, size=fbox, boxrule=1pt, pad at break*=1mm,colback=cellbackground, colframe=cellborder]
\prompt{In}{incolor}{5}{\boxspacing}
\begin{Verbatim}[commandchars=\\\{\}]
\PY{k+kn}{import} \PY{n+nn}{nltk} 
\PY{n}{entries} \PY{o}{=} \PY{n}{nltk}\PY{o}{.}\PY{n}{corpus}\PY{o}{.}\PY{n}{cmudict}\PY{o}{.}\PY{n}{entries}\PY{p}{(}\PY{p}{)}
\PY{n+nb}{len}\PY{p}{(}\PY{n}{entries}\PY{p}{)}
\end{Verbatim}
\end{tcolorbox}

            \begin{tcolorbox}[breakable, size=fbox, boxrule=.5pt, pad at break*=1mm, opacityfill=0]
\prompt{Out}{outcolor}{5}{\boxspacing}
\begin{Verbatim}[commandchars=\\\{\}]
133737
\end{Verbatim}
\end{tcolorbox}
        
    \begin{tcolorbox}[breakable, size=fbox, boxrule=1pt, pad at break*=1mm,colback=cellbackground, colframe=cellborder]
\prompt{In}{incolor}{6}{\boxspacing}
\begin{Verbatim}[commandchars=\\\{\}]
\PY{k}{for} \PY{n}{entry} \PY{o+ow}{in} \PY{n}{entries} \PY{p}{[}\PY{l+m+mi}{10000}\PY{p}{:}\PY{l+m+mi}{10025}\PY{p}{]}\PY{p}{:}
    \PY{n+nb}{print}\PY{p}{(}\PY{n}{entry}\PY{p}{)}
\end{Verbatim}
\end{tcolorbox}

    \begin{Verbatim}[commandchars=\\\{\}]
('belford', ['B', 'EH1', 'L', 'F', 'ER0', 'D'])
('belfry', ['B', 'EH1', 'L', 'F', 'R', 'IY0'])
('belgacom', ['B', 'EH1', 'L', 'G', 'AH0', 'K', 'AA0', 'M'])
('belgacom', ['B', 'EH1', 'L', 'JH', 'AH0', 'K', 'AA0', 'M'])
('belgard', ['B', 'EH0', 'L', 'G', 'AA1', 'R', 'D'])
('belgarde', ['B', 'EH0', 'L', 'G', 'AA1', 'R', 'D', 'IY0'])
('belge', ['B', 'EH1', 'L', 'JH', 'IY0'])
('belger', ['B', 'EH1', 'L', 'G', 'ER0'])
('belgian', ['B', 'EH1', 'L', 'JH', 'AH0', 'N'])
('belgians', ['B', 'EH1', 'L', 'JH', 'AH0', 'N', 'Z'])
('belgique', ['B', 'EH0', 'L', 'ZH', 'IY1', 'K'])
("belgique's", ['B', 'EH0', 'L', 'JH', 'IY1', 'K', 'S'])
('belgium', ['B', 'EH1', 'L', 'JH', 'AH0', 'M'])
("belgium's", ['B', 'EH1', 'L', 'JH', 'AH0', 'M', 'Z'])
('belgo', ['B', 'EH1', 'L', 'G', 'OW2'])
('belgrade', ['B', 'EH1', 'L', 'G', 'R', 'EY0', 'D'])
('belgrade', ['B', 'EH1', 'L', 'G', 'R', 'AA2', 'D'])
("belgrade's", ['B', 'EH1', 'L', 'G', 'R', 'EY0', 'D', 'Z'])
("belgrade's", ['B', 'EH1', 'L', 'G', 'R', 'AA2', 'D', 'Z'])
('belgrave', ['B', 'EH1', 'L', 'G', 'R', 'EY2', 'V'])
('beli', ['B', 'EH1', 'L', 'IY0'])
('belich', ['B', 'EH1', 'L', 'IH0', 'K'])
('belie', ['B', 'IH0', 'L', 'AY1'])
('belied', ['B', 'IH0', 'L', 'AY1', 'D'])
('belief', ['B', 'IH0', 'L', 'IY1', 'F'])
    \end{Verbatim}

    \begin{tcolorbox}[breakable, size=fbox, boxrule=1pt, pad at break*=1mm,colback=cellbackground, colframe=cellborder]
\prompt{In}{incolor}{7}{\boxspacing}
\begin{Verbatim}[commandchars=\\\{\}]
\PY{k+kn}{from} \PY{n+nn}{nltk}\PY{n+nn}{.}\PY{n+nn}{corpus} \PY{k}{import} \PY{n}{wordnet} \PY{k}{as} \PY{n}{wn}
\PY{n}{wn}\PY{o}{.}\PY{n}{synsets}\PY{p}{(}\PY{l+s+s2}{\PYZdq{}}\PY{l+s+s2}{phone}\PY{l+s+s2}{\PYZdq{}}\PY{p}{)} \PY{c+c1}{\PYZsh{} you get an id for subsets.}
\PY{n}{wn}\PY{o}{.}\PY{n}{synsets}\PY{p}{(}\PY{l+s+s2}{\PYZdq{}}\PY{l+s+s2}{house}\PY{l+s+s2}{\PYZdq{}}\PY{p}{)}
\end{Verbatim}
\end{tcolorbox}

            \begin{tcolorbox}[breakable, size=fbox, boxrule=.5pt, pad at break*=1mm, opacityfill=0]
\prompt{Out}{outcolor}{7}{\boxspacing}
\begin{Verbatim}[commandchars=\\\{\}]
[Synset('house.n.01'),
 Synset('firm.n.01'),
 Synset('house.n.03'),
 Synset('house.n.04'),
 Synset('house.n.05'),
 Synset('house.n.06'),
 Synset('house.n.07'),
 Synset('sign\_of\_the\_zodiac.n.01'),
 Synset('house.n.09'),
 Synset('family.n.01'),
 Synset('theater.n.01'),
 Synset('house.n.12'),
 Synset('house.v.01'),
 Synset('house.v.02')]
\end{Verbatim}
\end{tcolorbox}
        
    \begin{tcolorbox}[breakable, size=fbox, boxrule=1pt, pad at break*=1mm,colback=cellbackground, colframe=cellborder]
\prompt{In}{incolor}{8}{\boxspacing}
\begin{Verbatim}[commandchars=\\\{\}]
\PY{n}{wn}\PY{o}{.}\PY{n}{synset}\PY{p}{(}\PY{l+s+s1}{\PYZsq{}}\PY{l+s+s1}{call.v.03}\PY{l+s+s1}{\PYZsq{}}\PY{p}{)}\PY{o}{.}\PY{n}{lemma\PYZus{}names}\PY{p}{(}\PY{p}{)}
\end{Verbatim}
\end{tcolorbox}

            \begin{tcolorbox}[breakable, size=fbox, boxrule=.5pt, pad at break*=1mm, opacityfill=0]
\prompt{Out}{outcolor}{8}{\boxspacing}
\begin{Verbatim}[commandchars=\\\{\}]
['call', 'telephone', 'call\_up', 'phone', 'ring']
\end{Verbatim}
\end{tcolorbox}
        
    \begin{tcolorbox}[breakable, size=fbox, boxrule=1pt, pad at break*=1mm,colback=cellbackground, colframe=cellborder]
\prompt{In}{incolor}{9}{\boxspacing}
\begin{Verbatim}[commandchars=\\\{\}]
\PY{n}{wn}\PY{o}{.}\PY{n}{synset}\PY{p}{(}\PY{l+s+s1}{\PYZsq{}}\PY{l+s+s1}{telephone.n.01}\PY{l+s+s1}{\PYZsq{}}\PY{p}{)}\PY{o}{.}\PY{n}{lemma\PYZus{}names}\PY{p}{(}\PY{p}{)}
\end{Verbatim}
\end{tcolorbox}

            \begin{tcolorbox}[breakable, size=fbox, boxrule=.5pt, pad at break*=1mm, opacityfill=0]
\prompt{Out}{outcolor}{9}{\boxspacing}
\begin{Verbatim}[commandchars=\\\{\}]
['telephone', 'phone', 'telephone\_set']
\end{Verbatim}
\end{tcolorbox}
        
    \begin{tcolorbox}[breakable, size=fbox, boxrule=1pt, pad at break*=1mm,colback=cellbackground, colframe=cellborder]
\prompt{In}{incolor}{10}{\boxspacing}
\begin{Verbatim}[commandchars=\\\{\}]
\PY{n}{wn}\PY{o}{.}\PY{n}{synset}\PY{p}{(}\PY{l+s+s1}{\PYZsq{}}\PY{l+s+s1}{family.n.01}\PY{l+s+s1}{\PYZsq{}}\PY{p}{)}\PY{o}{.}\PY{n}{lemma\PYZus{}names}\PY{p}{(}\PY{p}{)}
\end{Verbatim}
\end{tcolorbox}

            \begin{tcolorbox}[breakable, size=fbox, boxrule=.5pt, pad at break*=1mm, opacityfill=0]
\prompt{Out}{outcolor}{10}{\boxspacing}
\begin{Verbatim}[commandchars=\\\{\}]
['family', 'household', 'house', 'home', 'menage']
\end{Verbatim}
\end{tcolorbox}
        
    \begin{tcolorbox}[breakable, size=fbox, boxrule=1pt, pad at break*=1mm,colback=cellbackground, colframe=cellborder]
\prompt{In}{incolor}{11}{\boxspacing}
\begin{Verbatim}[commandchars=\\\{\}]
\PY{n}{texts} \PY{o}{=} \PY{p}{[}\PY{l+s+s2}{\PYZdq{}\PYZdq{}\PYZdq{}}\PY{l+s+s2}{ Christopher Hemsworth born 11 August 1983 is an Australian actor. He rose to prominence playing Kim Hyde in the Australian TV series Home and Away (2004–07) }
\PY{l+s+s2}{before beginning a film career in Hollywood by taking on parts in the science fiction film Star Trek (2009) and the thriller A Perfect Getaway (2009).}
\PY{l+s+s2}{Hemsworth went on to star in the fantasy film Snow White and the Huntsman (2012), the war film Red Dawn (2012), the action thriller Blackhat (2015), }
\PY{l+s+s2}{the biographical thriller In the Heart of the Sea (2015),}
\PY{l+s+s2}{the comedy Ghostbusters (2016), and the Men in Black film series spin\PYZhy{}off Men in Black: International (2019). }
\PY{l+s+s2}{His most critically acclaimed roles include the comedy horror The Cabin in the Woods (2012) and the biographical sports film Rush (2013), in which he portrayed James Hunt.}
\PY{l+s+s2}{\PYZdq{}\PYZdq{}\PYZdq{}}\PY{p}{]}

\PY{c+c1}{\PYZsh{}the text from wikipedia about chris hemsworth.}
\PY{k}{for} \PY{n}{text} \PY{o+ow}{in} \PY{n}{texts}\PY{p}{:}
    \PY{n}{sentences} \PY{o}{=}\PY{n}{nltk}\PY{o}{.}\PY{n}{sent\PYZus{}tokenize}\PY{p}{(}\PY{n}{text}\PY{p}{)}
    \PY{k}{for} \PY{n}{sentence} \PY{o+ow}{in} \PY{n}{sentences}\PY{p}{:}
        \PY{n}{words} \PY{o}{=}\PY{n}{nltk}\PY{o}{.}\PY{n}{word\PYZus{}tokenize}\PY{p}{(}\PY{n}{sentence}\PY{p}{)}
        \PY{n}{tagged\PYZus{}words} \PY{o}{=}\PY{n}{nltk}\PY{o}{.}\PY{n}{pos\PYZus{}tag}\PY{p}{(}\PY{n}{words}\PY{p}{)}
        \PY{n+nb}{print}\PY{p}{(}\PY{n}{tagged\PYZus{}words}\PY{p}{)}
\end{Verbatim}
\end{tcolorbox}

    \begin{Verbatim}[commandchars=\\\{\}]
[('Christopher', 'NNP'), ('Hemsworth', 'NNP'), ('born', 'VBD'), ('11', 'CD'),
('August', 'NNP'), ('1983', 'CD'), ('is', 'VBZ'), ('an', 'DT'), ('Australian',
'JJ'), ('actor', 'NN'), ('.', '.')]
[('He', 'PRP'), ('rose', 'VBD'), ('to', 'TO'), ('prominence', 'VB'), ('playing',
'VBG'), ('Kim', 'NNP'), ('Hyde', 'NNP'), ('in', 'IN'), ('the', 'DT'),
('Australian', 'JJ'), ('TV', 'NN'), ('series', 'NN'), ('Home', 'NNP'), ('and',
'CC'), ('Away', 'NNP'), ('(', '('), ('2004–07', 'CD'), (')', ')'), ('before',
'IN'), ('beginning', 'VBG'), ('a', 'DT'), ('film', 'NN'), ('career', 'NN'),
('in', 'IN'), ('Hollywood', 'NNP'), ('by', 'IN'), ('taking', 'VBG'), ('on',
'IN'), ('parts', 'NNS'), ('in', 'IN'), ('the', 'DT'), ('science', 'NN'),
('fiction', 'NN'), ('film', 'NN'), ('Star', 'NNP'), ('Trek', 'NNP'), ('(', '('),
('2009', 'CD'), (')', ')'), ('and', 'CC'), ('the', 'DT'), ('thriller', 'NN'),
('A', 'NNP'), ('Perfect', 'NNP'), ('Getaway', 'NNP'), ('(', '('), ('2009',
'CD'), (')', ')'), ('.', '.')]
[('Hemsworth', 'NNP'), ('went', 'VBD'), ('on', 'IN'), ('to', 'TO'), ('star',
'VB'), ('in', 'IN'), ('the', 'DT'), ('fantasy', 'NN'), ('film', 'NN'), ('Snow',
'NNP'), ('White', 'NNP'), ('and', 'CC'), ('the', 'DT'), ('Huntsman', 'NNP'),
('(', '('), ('2012', 'CD'), (')', ')'), (',', ','), ('the', 'DT'), ('war',
'NN'), ('film', 'NN'), ('Red', 'NNP'), ('Dawn', 'NNP'), ('(', '('), ('2012',
'CD'), (')', ')'), (',', ','), ('the', 'DT'), ('action', 'NN'), ('thriller',
'NN'), ('Blackhat', 'NNP'), ('(', '('), ('2015', 'CD'), (')', ')'), (',', ','),
('the', 'DT'), ('biographical', 'JJ'), ('thriller', 'NN'), ('In', 'IN'), ('the',
'DT'), ('Heart', 'NNP'), ('of', 'IN'), ('the', 'DT'), ('Sea', 'NNP'), ('(',
'('), ('2015', 'CD'), (')', ')'), (',', ','), ('the', 'DT'), ('comedy', 'NN'),
('Ghostbusters', 'NNP'), ('(', '('), ('2016', 'CD'), (')', ')'), (',', ','),
('and', 'CC'), ('the', 'DT'), ('Men', 'NNP'), ('in', 'IN'), ('Black', 'NNP'),
('film', 'NN'), ('series', 'NN'), ('spin-off', 'NN'), ('Men', 'NNP'), ('in',
'IN'), ('Black', 'NNP'), (':', ':'), ('International', 'NNP'), ('(', '('),
('2019', 'CD'), (')', ')'), ('.', '.')]
[('His', 'PRP\$'), ('most', 'RBS'), ('critically', 'RB'), ('acclaimed', 'JJ'),
('roles', 'NNS'), ('include', 'VBP'), ('the', 'DT'), ('comedy', 'NN'),
('horror', 'VBD'), ('The', 'DT'), ('Cabin', 'NNP'), ('in', 'IN'), ('the', 'DT'),
('Woods', 'NNP'), ('(', '('), ('2012', 'CD'), (')', ')'), ('and', 'CC'), ('the',
'DT'), ('biographical', 'JJ'), ('sports', 'NNS'), ('film', 'NN'), ('Rush',
'NNP'), ('(', '('), ('2013', 'CD'), (')', ')'), (',', ','), ('in', 'IN'),
('which', 'WDT'), ('he', 'PRP'), ('portrayed', 'VBD'), ('James', 'NNP'),
('Hunt', 'NNP'), ('.', '.')]
    \end{Verbatim}

    \begin{tcolorbox}[breakable, size=fbox, boxrule=1pt, pad at break*=1mm,colback=cellbackground, colframe=cellborder]
\prompt{In}{incolor}{12}{\boxspacing}
\begin{Verbatim}[commandchars=\\\{\}]
\PY{k+kn}{import} \PY{n+nn}{nltk}
\PY{k+kn}{from} \PY{n+nn}{nltk}\PY{n+nn}{.}\PY{n+nn}{tokenize} \PY{k}{import} \PY{n}{TweetTokenizer}
\PY{n}{text} \PY{o}{=}\PY{l+s+s1}{\PYZsq{}}\PY{l+s+s1}{ After years of rebuilding OTHER nations, we are finally rebuilding OUR nation. We are finally putting AMERICA FIRST! :) }\PY{l+s+s1}{\PYZsq{}} \PY{c+c1}{\PYZsh{}Donald J trump Tweet}
\PY{n}{twtkn} \PY{o}{=}\PY{n}{TweetTokenizer}\PY{p}{(}\PY{p}{)}
\PY{n}{twtkn}\PY{o}{.}\PY{n}{tokenize}\PY{p}{(}\PY{n}{text}\PY{p}{)}
\end{Verbatim}
\end{tcolorbox}

            \begin{tcolorbox}[breakable, size=fbox, boxrule=.5pt, pad at break*=1mm, opacityfill=0]
\prompt{Out}{outcolor}{12}{\boxspacing}
\begin{Verbatim}[commandchars=\\\{\}]
['After',
 'years',
 'of',
 'rebuilding',
 'OTHER',
 'nations',
 ',',
 'we',
 'are',
 'finally',
 'rebuilding',
 'OUR',
 'nation',
 '.',
 'We',
 'are',
 'finally',
 'putting',
 'AMERICA',
 'FIRST',
 '!',
 ':)']
\end{Verbatim}
\end{tcolorbox}
        
    \begin{tcolorbox}[breakable, size=fbox, boxrule=1pt, pad at break*=1mm,colback=cellbackground, colframe=cellborder]
\prompt{In}{incolor}{13}{\boxspacing}
\begin{Verbatim}[commandchars=\\\{\}]
\PY{k+kn}{from} \PY{n+nn}{nltk}\PY{n+nn}{.}\PY{n+nn}{corpus} \PY{k}{import} \PY{n}{brown}
\PY{n}{news\PYZus{}text} \PY{o}{=}\PY{n}{brown}\PY{o}{.}\PY{n}{words}\PY{p}{(}\PY{n}{categories} \PY{o}{=}\PY{l+s+s1}{\PYZsq{}}\PY{l+s+s1}{news}\PY{l+s+s1}{\PYZsq{}}\PY{p}{)}
\PY{n}{fdist} \PY{o}{=}\PY{n}{nltk}\PY{o}{.}\PY{n}{FreqDist}\PY{p}{(}\PY{n}{w}\PY{o}{.}\PY{n}{lower}\PY{p}{(}\PY{p}{)} \PY{k}{for} \PY{n}{w} \PY{o+ow}{in} \PY{n}{news\PYZus{}text}\PY{p}{)}
\PY{n}{modals} \PY{o}{=}\PY{p}{[}\PY{l+s+s1}{\PYZsq{}}\PY{l+s+s1}{can}\PY{l+s+s1}{\PYZsq{}}\PY{p}{,}\PY{l+s+s1}{\PYZsq{}}\PY{l+s+s1}{could}\PY{l+s+s1}{\PYZsq{}}\PY{p}{,}\PY{l+s+s1}{\PYZsq{}}\PY{l+s+s1}{may}\PY{l+s+s1}{\PYZsq{}}\PY{p}{,}\PY{l+s+s1}{\PYZsq{}}\PY{l+s+s1}{might}\PY{l+s+s1}{\PYZsq{}}\PY{p}{,}\PY{l+s+s1}{\PYZsq{}}\PY{l+s+s1}{must}\PY{l+s+s1}{\PYZsq{}}\PY{p}{,}\PY{l+s+s1}{\PYZsq{}}\PY{l+s+s1}{will}\PY{l+s+s1}{\PYZsq{}}\PY{p}{,}\PY{l+s+s1}{\PYZsq{}}\PY{l+s+s1}{what}\PY{l+s+s1}{\PYZsq{}}\PY{p}{]}
\PY{k}{for} \PY{n}{m} \PY{o+ow}{in} \PY{n}{modals}\PY{p}{:}
    \PY{n+nb}{print}\PY{p}{(}\PY{n}{m}\PY{o}{+} \PY{l+s+s1}{\PYZsq{}}\PY{l+s+s1}{:}\PY{l+s+s1}{\PYZsq{}}\PY{p}{,}\PY{n}{fdist}\PY{p}{[}\PY{n}{m}\PY{p}{]}\PY{p}{,}\PY{n}{end}\PY{o}{=}\PY{l+s+s1}{\PYZsq{}}\PY{l+s+se}{\PYZbs{}n}\PY{l+s+s1}{ }\PY{l+s+s1}{\PYZsq{}}\PY{p}{)}
\end{Verbatim}
\end{tcolorbox}

    \begin{Verbatim}[commandchars=\\\{\}]
can: 94
 could: 87
 may: 93
 might: 38
 must: 53
 will: 389
 what: 95
    \end{Verbatim}

    \begin{tcolorbox}[breakable, size=fbox, boxrule=1pt, pad at break*=1mm,colback=cellbackground, colframe=cellborder]
\prompt{In}{incolor}{ }{\boxspacing}
\begin{Verbatim}[commandchars=\\\{\}]

\end{Verbatim}
\end{tcolorbox}


    % Add a bibliography block to the postdoc
    
    
    
\end{document}
